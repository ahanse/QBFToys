\documentclass{scrartcl}

\usepackage[utf8]{inputenc}
\usepackage[english]{babel}
\usepackage{amsmath, amsfonts, amsthm}

\DeclareMathOperator{\quant}{quant} 
\DeclareMathOperator{\var}{var} 
\newtheorem{definition}{Definition}

\title{Notes on QBF preprocessing techniques}

\begin{document}
\maketitle

This document tries to give a short overview over preprocessing techniques for
QBF problems, used by \emph{bloqqer} and other QBF preprocessors. It might be
interesting to investigate QBF systems, which are able to handle problems not
given in clause normal form.

Most of the notes here are based on \cite{biere11}.

\section{Techniques}
In the following text a QBF problem is assumed to be in quantified clause normal
form. The problems are denoted as $\phi = S_1\dots S_n\psi$. $\psi$ is said to
be the matrix and $S_i$ is a quantifier followed by a list of variables. The
variable lists are pairwise disjunct.
$\quant(l)\in\{\forall,\exists\}$ is the quantifier assigned to the literal
$l$ by some $S_i$. Given a literal $l$, $\bar{l}$ denotes the negation of $l$.
$S_1\dots S_n$ is is referred to as the $\emph{quantifier prefix}$.
For two literals $l,k$ the $l\leq k$ denotes that the variable of $l$ is
quantified earlier than the variable of $k$ ($\var(l)\in S_i, \var(k)\in S_j$
with $i\leq j$)

\subsection{Forall Reduction}
Forall reduction by itself can hardly be called a preprocessing technique, but
is part of the subsequent mentioned techniques. 

Given two clauses $C_1$ and $C_2$ with $l\in C_1$, $\bar{l}\in C_2$ and
$\quant(l)=\exists$, the \emph{Q-resolvent} $C_1\otimes C_2$ is defined
as $C_1'\setminus\{l\}\cup C_2'\setminus\{\bar{l}\}$.

$C_i'$ is the forall reduced clause $C_i$.
\begin{equation*}
C_i'=C_i\setminus\left\{k | k\in C_i,\quant(k)=\forall,\forall k'\in C_i\text{ with }
\quant(k')=\exists : k>k'\right\}
\end{equation*}

\subsection{Blocked Clause Elimination}
This is the primary technique of \emph{bloqqer}. 
\begin{definition}{Quantified Blocking Literal}
$l$ appearing in a clause $C$, is a \emph{quantified blocking literal} if
$\quant(l)=\exists$ and if forall other clauses $C'$ with $\bar{l}\in C'$,
there exists a literal $k$, $k\leq l$, s.t.  $k,\bar{k}\in C\otimes C'$.
\end{definition}

A clause is then said to be \emph{quantified blocked} if it contains a
quantified blocking literal. Quantified blocked clauses can be removed from a
QBF problem, without changing the truth value of the problem. 
Quantified blocked clause elimination (QBCE) thus removes quantified blocked clauses
until no more clauses are quantified blocked. 

The paper gives an in depth discussion of QBCE. The paper proves, that some
earlier preprocessing techniques are subsumed by QBCE.

\subsection{Quantified Blocked Covered Clause Elimination}
The \emph{bloqqer} tool allows the user to (de)activate the \emph{Covered
Literal Addition} technique. This technique is part of quantified blocked
covered clause elimination, which is an extension to QBCE.

The paper states: 
\begin{quote}\textbf{Quantified Blocked Covered Clause Elimination}
Covered clauses are clauses which are blocked or tautological when they are
enriched with literals contained in any resolvent with pivot element $l$, the
covering literal.
\end{quote}
A formal definition can be found in the paper.
Enriching a clause with the literals as noted in the quote, does not change
unsatisfiability. If the resulting clause is either blocked or tautological, it
can be freely removed. 

\subsection{Hidden Blocked Clause Elimination/Tautology Elimination}

Using implications contained within a QBF literals are added to clauses, to
discover blocked clauses and/or tautologies.

A literal is a \emph{quantified hidden literal} with respect to some clause $C$.
\begin{definition}{Quantified Hidden Literal}
Let $\phi=S_1\dots S_n\psi$ be a QBF and $C$ be a clause in $\psi$. A literal
$l$ is a \emph{quantified hidden literal}, if the QBF problem $\phi$ has
contains a clause $(l_1,\dots,l_n,\bar{l})$ with $l_i\leq l$ and
$l_1,\dots,\l_n\in C$.
\end{definition}

If $l$ is a quantified hidden literal with respect to some clause $C$, then this
clause can be replaced with $C\cup\{l\}$, without changing the unsatisfiability
of the QBF problem.

Again clauses obtained by adding quantified hidden literals might become blocked
or tautological and thus might be removed safely. 

\subsection{Hidden Clause Elimination}

\subsection{Equivalent Literal Reasoning}
\subsection{Variable Elimination}
\subsection{Variable Expansion}
    
\begin{thebibliography}{9}
\bibitem{biere11}
Armin Biere, Florian Lonsing, and Martina Seidl,
\emph{Blocked Clause Elimination for QBF},
CADE 2011,
pp. 101-115,
Springer-Verlag Berlin Heidelberg.
\end{thebibliography}
\end{document}
